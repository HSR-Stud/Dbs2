\section{Indizes}


\subsection{Vergleich von Indizes}

\begin{tabular}{*4{l|}}
	\cline{2-4}
	& \textbf{B-Tree} & \textbf{Bitmap} & \textbf{Hash Table} \\
	\hline
	\multicolumn{1}{|l|}{Wertebereich Spalten} & Gross & Klein & x \\
	\multicolumn{1}{|l|}{Arten von Abfragen} & Range- und Equal-Search & x & Equal-Search \\
	\multicolumn{1}{|l|}{Effizienz von Updates} & Gut & Schlecht & x \\
	\hline
\end{tabular}


\subsection{Clustered und Unclustered Indizes}

\textbf{Clustered Index}

Blätter enthalten direkt die Data-Records in sortierter Reihenfolge. Die Daten
liegen so physikalisch nahe beieinander, was den Zugriff beschleunigt. Pro
Datenbanktabelle kann es dadurch aber nur einen Clustered Index geben.

Wird oft für Primärschlüssel verwendet.

\textbf{Unclustered Index}

Blätter enthalten Referenzen auf die unsortierten Records. So sind pro Tabelle
mehrere Unclustered Indizes möglich.

Wird oft für non-PK-Spalten verwendet welche häufig in JOIN, WHERE oder ORDER BY
Queries vorkommen.


\subsection{B-Trees}

B-Bäume eignen sich -- im Gegensatz zu den konventionellen Binärbäumen -- gut
für Sekundärspeicher. Die Stärken liegen bei Range- und Equal-Search-Queries.
Das Laufzeitverhalten für Suche, sequentiellen Zugriff, Inserts und Deletes
ist logarithmisch.

Ein B-Baum mit Grad $k$ hat folgende Eigenschaften:

\begin{enumerate}
	\item Jeder Weg von der Wurzel zu einem Blatt hat die gleiche Länge.
	\item Jeder Knoten ausser der Wurzel hat mindestens $k$ und höchstens $2k$
		Einträge. Die Wurzel hat zwischen einem und $2k$ Einträgen. Die Einträge
		werden in allen Knoten sortiert gehalten.
	\item Alle Knoten mit $n$ Einträgen, ausser den Blättern, haben $n+1$ Kinder.
	\item Seien $S_1, \ldots, S_n$ die Schlüssel eines Knotens mit $n+1$ Kindern.
		$V_0, V_1, \ldots, V_n$ seien die Verweise auf diese Kinder. Dann gilt:
		\begin{itemize}
			\item $V_0$ weist auf den Teilbaum mit Schlüsseln kleiner als $S_1$.
			\item $V_1$ ($i=1, \ldots, n-1$) weist auf den Teilbaum, dessen Schlüssel
				zwischen $S_i$ und $S_i+1$ liegen.
			\item $V_n$ weist auf den Teilbaum mit Schlüsseln grösser als $S_n$.
			\item In den Blattknoten sind die Zeiger nicht definiert.
		\end{itemize}
\end{enumerate}


\subsection{Bitmap}

Ein Bitmap-Index ordnet Attributwerte als Bitmuster.

Gut geeignet für Attribute mit wenigen diskreten Werten, d.h. Wertebereiche mit
geringer Kardinalität (ca. 1\% der Datenmenge).

Anfragen (v.a. AND/OR) sind schnell, Modifikationen sehr langsam. Dadurch sind
Bitmap Indizes gut für Datawarehouses geeignet, wo der Zugriff nur read-only
erfolgt.


\subsection{Hash}

Gut geeignet für für Equal-Search (zB Equi-Joins) und Multipoint Queries.

Schlecht geeignet für Range-, Prefix- oder Extremal-Queries.

NULL-Werte sind nicht zugelassen.

Probleme: Hash overflows. Diese können mit Overflow Chains verhindert werden.
